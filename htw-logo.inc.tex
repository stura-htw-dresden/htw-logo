% Zur direkten Verwendung mit LaTeX - z.B. in Arbeiten an der/über die HTW selbst
% mit
%% Zur direkten Verwendung mit LaTeX - z.B. in Arbeiten an der/über die HTW selbst
% mit
%% Zur direkten Verwendung mit LaTeX - z.B. in Arbeiten an der/über die HTW selbst
% mit
%% Zur direkten Verwendung mit LaTeX - z.B. in Arbeiten an der/über die HTW selbst
% mit
%\input{htw-logo.inc}
% s.a. Beispiel htw-logo.demo.tex

\usepackage{tikz}

% rgb: die Farbe wird als dreistelliger Vektor angegeben. Die Komponenten erhalten Werte zwischen 0 und 1 und stehen für rot, grün und blau.
%~ \definecolor{schwarz}{RGB}{0,0,0}
%~ \definecolor{htworange}{RGB}{240,172,43}
% cmyk: die Farbe wird als vierstelliger Vektor angegeben. Die Komponenten erhalten Werte zwischen 0 und 1 und stehen für cyan, magenta, gelb und schwarz.
\definecolor{schwarz}{cmyk}{0,0,0,100}
\definecolor{htworange}{cmyk}{0,.41,.94,.03}

% HTW Logo mit TikZ in Makro
\newcommand{\htwlogo}{
	\begin{tikzpicture}[scale=.01]
		\path[fill=schwarz] (0,0) -- (0,146) -- (29.5,146) -- (29.5,76) -- (88.5,76) -- (88.5,146) -- (159,146) -- (159,0) -- (130,0) -- (130,140) -- (94.5,140) -- (94.5,0) -- (88.5,0) -- (88.5,70) -- (29.5,70) -- (29.5,0) -- cycle;
		\path[fill=schwarz] (189,0) -- (189,146) -- (233,146) -- (233,140) -- (195,140) -- (195,10.5) -- (271.5,146) -- (303,146) -- (220.5,0) -- cycle;
		\path[fill=htworange] (253.5,0) -- (336,146) -- (367,146) -- (285.5,0) -- cycle;
	\end{tikzpicture}
}

% s.a. Beispiel htw-logo.demo.tex

\usepackage{tikz}

% rgb: die Farbe wird als dreistelliger Vektor angegeben. Die Komponenten erhalten Werte zwischen 0 und 1 und stehen für rot, grün und blau.
%~ \definecolor{schwarz}{RGB}{0,0,0}
%~ \definecolor{htworange}{RGB}{240,172,43}
% cmyk: die Farbe wird als vierstelliger Vektor angegeben. Die Komponenten erhalten Werte zwischen 0 und 1 und stehen für cyan, magenta, gelb und schwarz.
\definecolor{schwarz}{cmyk}{0,0,0,100}
\definecolor{htworange}{cmyk}{0,.41,.94,.03}

% HTW Logo mit TikZ in Makro
\newcommand{\htwlogo}{
	\begin{tikzpicture}[scale=.01]
		\path[fill=schwarz] (0,0) -- (0,146) -- (29.5,146) -- (29.5,76) -- (88.5,76) -- (88.5,146) -- (159,146) -- (159,0) -- (130,0) -- (130,140) -- (94.5,140) -- (94.5,0) -- (88.5,0) -- (88.5,70) -- (29.5,70) -- (29.5,0) -- cycle;
		\path[fill=schwarz] (189,0) -- (189,146) -- (233,146) -- (233,140) -- (195,140) -- (195,10.5) -- (271.5,146) -- (303,146) -- (220.5,0) -- cycle;
		\path[fill=htworange] (253.5,0) -- (336,146) -- (367,146) -- (285.5,0) -- cycle;
	\end{tikzpicture}
}

% s.a. Beispiel htw-logo.demo.tex

\usepackage{tikz}

% rgb: die Farbe wird als dreistelliger Vektor angegeben. Die Komponenten erhalten Werte zwischen 0 und 1 und stehen für rot, grün und blau.
%~ \definecolor{schwarz}{RGB}{0,0,0}
%~ \definecolor{htworange}{RGB}{240,172,43}
% cmyk: die Farbe wird als vierstelliger Vektor angegeben. Die Komponenten erhalten Werte zwischen 0 und 1 und stehen für cyan, magenta, gelb und schwarz.
\definecolor{schwarz}{cmyk}{0,0,0,100}
\definecolor{htworange}{cmyk}{0,.41,.94,.03}

% HTW Logo mit TikZ in Makro
\newcommand{\htwlogo}{
	\begin{tikzpicture}[scale=.01]
		\path[fill=schwarz] (0,0) -- (0,146) -- (29.5,146) -- (29.5,76) -- (88.5,76) -- (88.5,146) -- (159,146) -- (159,0) -- (130,0) -- (130,140) -- (94.5,140) -- (94.5,0) -- (88.5,0) -- (88.5,70) -- (29.5,70) -- (29.5,0) -- cycle;
		\path[fill=schwarz] (189,0) -- (189,146) -- (233,146) -- (233,140) -- (195,140) -- (195,10.5) -- (271.5,146) -- (303,146) -- (220.5,0) -- cycle;
		\path[fill=htworange] (253.5,0) -- (336,146) -- (367,146) -- (285.5,0) -- cycle;
	\end{tikzpicture}
}

% s.a. Beispiel htw-logo.demo.tex

\usepackage{tikz}
\usepackage{environ}

\makeatletter
\newsavebox{\measure@tikzpicture}
\NewEnviron{scaletikzpicturetowidth}[1]{%
  \def\tikz@width{#1}%
  \def\tikzscale{1}\begin{lrbox}{\measure@tikzpicture}%
  \BODY
  \end{lrbox}%
  \pgfmathparse{#1/\wd\measure@tikzpicture}%
  \edef\tikzscale{\pgfmathresult}%
  \BODY
}
\NewEnviron{scaletikzpicturetoheight}[1]{%
  \def\tikz@width{#1}%
  \def\tikzscale{1}\begin{lrbox}{\measure@tikzpicture}%
  \BODY
  \end{lrbox}%
  \pgfmathparse{#1/\ht\measure@tikzpicture}%
  \edef\tikzscale{\pgfmathresult}%
  \BODY
}
\makeatother

% Logo mit TikZ
\newcommand{\logo}{
    \begin{tikzpicture}[scale=\tikzscale]
        \draw[black,fill=black] (0,0) -- (0,146) -- (29.5,146) -- (29.5,76) -- (88.5,76) -- (88.5,146) -- (159,146) -- (159,0) -- (130,0) -- (130,140) -- (94.5,140) -- (94.5,0) -- (88.5,0) -- (88.5,70) -- (29.5,70) -- (29.5,0) -- cycle;
        \draw[black,fill=black] (189,0) -- (189,146) -- (233,146) -- (233,140) -- (195,140) -- (195,10.5) -- (271.5,146) -- (303,146) -- (220.5,0) -- cycle;
        \draw[orange,fill=orange] (253.5,0) -- (336,146) -- (367,146) -- (285.5,0) -- cycle;
    \end{tikzpicture}
}
